\newpage
\section{Przebieg gry}

Celem gracza w pojedynczym rozdaniu (nazywanym \emph{ręką}) jest złożenie
ręki.
Celem gry jest zebranie jak największej łącznej liczby punktów z wygrywających
rąk.
W szczególności nie ma znaczenia, ile rąk złoży gracz --- jedynym kryterium
wygranej jest jego końcowy wynik.

\subsection{Fazy gry}
Tura gracza rozpoczyna się w momencie pozyskania przez niego kamienia i kończy
w momencie odrzucenia kamienia.
Podczas zwyczajnej sekwencji tur każdy z graczy rozgrywa swoją turę dokładnie
raz.
Zwyczajna sekwencja zostaje przerwana, jeśli odrzucony kamień zostaje
zameldowany do \texttt{chii}, \texttt{pon} lub \texttt{kan} lub też jeżeli gracz
w trakcie swojej tury zamelduje \texttt{ankan} (zamknięty \texttt{kan}).
Rozdanie kończy się, gdy któryś z graczy złoży rękę (co daje mu wygraną), lub
też gdy zostanie ogłoszony remis.
Podczas pojedynczej rundy wszyscy gracze kolejno grają jako gracze Wschodu.
Kompletna gra składa się z dwóch rund: rundy wschodu i rundy południa.

\subsection{Ręka}
Kompletna ręka w mahjongu składa się z czterech \emph{grup} oraz pary.
Grupy dzielą się na sekwensy (\texttt{shuntsu}), trójki (\texttt{koutsu})
i czwórki (\texttt{kantsu}).
Dodatkowo, kompletna ręka musi mieć przynajmniej jedno \texttt{yaku} (punktowany
układ).
Gracz będący \texttt{furiten} nie może wygrać na kamieniu odrzuconym przez
innego gracza.

Sekwens to trzy kolejne kamienie liczbowe z tej samej rodziny --- sekwensów nie
można tworzyć ze smoków ani wiatrów.
Kamienie 8-9-1 oraz 9-1-2 z tej samej rodziny nie tworzą sekwensu.

Trójka składa się z trzech identycznych kamieni.

Czwórka składa się z czterech identycznych kamieni.

Para składa się z dwóch identycznych kamieni.

\begin{center}
    \begin{tabular}{lc}
        \raisebox{1.2em}{Sekwens (\texttt{shuntsu})}
            & \DrawHand{m456}{3em} \\
        \raisebox{1.2em}{Trójka (\texttt{koutsu})}
            & \DrawHand{p444}{3em} \\
        \raisebox{1.2em}{Czwórka (\texttt{kantsu})}
            & \DrawHand{z1111}{3em}
    \end{tabular}
\end{center}

W Riichi Mahjongu istnieją dwie specjalne ręce, które nie wymagają posiadania
czterech grup oraz pary: \texttt{chiitoitsu} oraz \texttt{kokushi musou}.
