\newpage
\section{Przebieg gry}

Celem gracza w pojedynczym rozdaniu (nazywanym \emph{ręką}) jest złożenie
ręki.
Celem gry jest zebranie jak największej łącznej liczby punktów z wygrywających
rąk.
W szczególności nie ma znaczenia, ile rąk złoży gracz --- jedynym kryterium
wygranej jest jego końcowy wynik.

\subsection{Fazy gry}
Tura gracza rozpoczyna się w momencie pozyskania przez niego kamienia i kończy
w momencie odrzucenia kamienia.
Podczas zwyczajnej sekwencji tur każdy z graczy rozgrywa swoją turę dokładnie
raz.
Zwyczajna sekwencja zostaje przerwana, jeśli odrzucony kamień zostaje skradziony
lub też jeżeli gracz w trakcie swojej tury zamelduje \texttt{ankan} (zamknięty
\texttt{kan}). Rozdanie kończy się, gdy któryś z graczy złoży rękę (co daje mu
wygraną), lub też gdy zostanie ogłoszony remis.
Podczas pojedynczej rundy wszyscy gracze kolejno grają jako gracze Wschodu.
Kompletna gra składa się z dwóch rund: rundy wschodu i rundy południa.

\subsection{Ręka}
Kompletna ręka w mahjongu składa się z czterech \emph{grup} oraz pary.
Grupy dzielą się na sekwensy (\texttt{shuntsu}), trójki (\texttt{koutsu})
i czwórki (\texttt{kantsu}).
Dodatkowo, kompletna ręka musi mieć przynajmniej jedno \texttt{yaku} (punktowany
układ).
Gracz będący \texttt{furiten} nie może wygrać na kamieniu odrzuconym przez
innego gracza.

Sekwens to trzy kolejne kamienie liczbowe z tej samej rodziny --- sekwensów nie
można tworzyć ze smoków ani wiatrów.
Kamienie 8-9-1 oraz 9-1-2 z tej samej rodziny nie tworzą sekwensu.

Trójka składa się z trzech identycznych kamieni.

Czwórka składa się z czterech identycznych kamieni.

Para składa się z dwóch identycznych kamieni.

\begin{center}
    \begin{tabular}{lc}
        \raisebox{1.2em}{Sekwens (\texttt{shuntsu})}
            & \DrawHand{m456}{3em} \\
        \raisebox{1.2em}{Trójka (\texttt{koutsu})}
            & \DrawHand{p444}{3em} \\
        \raisebox{1.2em}{Czwórka (\texttt{kantsu})}
            & \DrawHand{z1111}{3em}
    \end{tabular}
\end{center}

W Riichi Mahjongu istnieją dwie specjalne ręce, które nie wymagają posiadania
czterech grup oraz pary: \texttt{chiitoitsu} oraz \texttt{kokushi musou}.

\subsection{Tura gracza}
Gracze rozgrywają kolejno swoje tury.
Zaczyna gracz Wschodu, a po nim kolejno gracze przeciwnie do ruchu wskazówek
zegara.

Gracz rozpoczyna swoją turę dobraniem kamienia z muru.
Jednakże, ponieważ gracz Wschodu rozpoczyna rozdanie z czternastoma kamieniami,
podczas swojej pierwszej tury nie dobiera on kamienia.
Jeśli gracz nie może (lub nie chce) zadeklarować wygranej lub \texttt{kan},
kończy on swoją turę odrzucając jeden ze swoich zasłoniętych kamieni.
Podczas swojej pierwszej tury gracz Wschodu powinien przed odrzutem poczekać, aż
pozostali gracze zobaczą i posortują swoje kamienie.

Gracze powinni zwrócić uwagę na to, by odrzucać kamienie nie zasłaniając ich
ręką.
Odrzuty są umieszczane przed graczem wewnątrz muru w porządku od lewej do
prawej, po sześć kamieni w rzędzie --- tak, aby było klarowne, kto (i w jakiej
kolejności) odrzucił które kamienie.

\subsubsection{Pierwszeństwo i czas przy kradzieży kamieni}
% TODO: Przez dowolnego innego?
% https://discord.com/channels/591384413410164765/656637295491022848/850873563707473920
Ostatno odrzucony kamień może zostać przywłaszczony przez dowolnego gracza do
trójki lub czwórki, dopóki następny gracz nie dobierze kolejnego kamienia z muru.
Na ostatnio odrzuconym kamieniu można zadeklarować wygraną, dopóki następny
gracz nie wykona odrzutu --- wyjątkiem jest sytuacja, w której następny gracz
zadeklaruje wygraną przez dobór (\texttt{tsumo}).

Kradzież do trójki lub czwórki może skutkować utratą kolejki przez graczy,
ponieważ gra zaczyna toczyć się dalej zaczynając od gracza kradnącego, a nie
odrzucającego.
Jeżeli któryś z graczy zamelduje wygraną na odrzuconym kamieniu, wszystkie
równoczesne meldunki kradzieży tego kamienia zostają zignorowane.
Kilku graczy może wygrać na tym samym kamieniu jednocześnie.

Gracz, którego tura ma nastąpić bezpośrednio po odrzucie kamienia, może ukraść
ten kamień do sekwensu.
Jeśli gracz nie zdecyduje się na kradzież, rozpoczyna swoją turę zwyczajnie
dobierając kamień z muru.
\begin{figure}[h]
    \centering
    \includesvg[width=0.7\linewidth]{images/table-midgame.svg}
    \caption{Przykładowy stan stołu w trakcie gry.}
    \label{fig:table-midgame}
\end{figure}

Deklaracja wygranej na odrzuconym kamieniu ma pierwszeństwo przed każdą inną
deklaracją dotyczącą tego kamienia.
Kradzież do trójki lub czwórki ma pierwszeństwo przed kradzieżą do sekwensu.
Gracz, który zadeklarował wygraną, nie może zmienić swojej deklaracji.

Gracze nie mają limitu czasowego na zagrania, ale oczekuje się od nich gry w
rozsądnym tempie.
Gracz dobierający kamień zbyt szybko tak, że inni gracze nie mają czasu wykonać
meldunków, lub wielokrotnie zbytnio przedłużający swoją turę może być decyzją
sędziego ukarany za utrudnianie gry.

Jeśli następny gracz dobierze kamień zbyt szybko uniemożliwiając innemu graczowi
wykonanie meldunku, meldunek uznaje się za ważny, a dobrany kamień umieszcza się
z powrotem w murze.

\subsubsection{Kradzież z podmianą (\texttt{kuikae})}
Kradzież z podmianą (\emph{swap-calling}, \texttt{kuikae}) nie jest dozwolona:
nie jest dozwolona kradzież kamienia i natychmiastowy odrzut tego samego
kamienia.
Ponadto nie jest dozwolona kradzież kamienia do sekwensu i odrzut kamienia z
drugiego końca sekwensu.
\begin{center}
    \begin{tabular}{cccc}
        \raisebox{0.7em}{Nie można ukraść\quad}\DrawHand{p1}{2em}
            & \raisebox{0.7em}{do trójki}
            & \DrawHand{p11*1}{2em}\raisebox{0.7em}{, a następnie odrzucić}
            & \DrawHand{p1}{2em}\raisebox{0.7em}{.} \\
        \raisebox{0.7em}{Nie można ukraść\quad}\DrawHand{p1}{2em}
            & \raisebox{0.7em}{do sekwensu}
            & \DrawHand{p1*23}{2em}\raisebox{0.7em}{, a następnie odrzucić}
            & \DrawHand{p1}{2em}\raisebox{0.7em}{ lub }\DrawHand{p4}{2em}\raisebox{0.7em}{.}
    \end{tabular}
\end{center}

\subsubsection{Otwarty sekwens}
Kamień do sekwensu może być skradziony wyłącznie od gracza po lewej.
Kradzież ostatnio odrzuconego kamienia do sekwensu rozpoczyna się od wyraźnego
powiedzenia na głos ,,chow'' lub ,,chii''.
Następnie gracz odsłania pasujące kamienie ze swojej ręki, po czym odrzuca
kamień ze swojej ręki i zabiera skradzony kamień.
Kolejność dwóch ostatnich czynności nie jest istotna: gracz może najpierw zabrać
odrzucony kamień, a dopiero potem wykonać odrzut.

\subsubsection{Otwarta trójka}
Kradzież ostatnio odrzuconego kamienia do trójki rozpoczyna się od wyraźnego
powiedzenia na głos ,,pung'' lub ,,pon''.
Następnie gracz odsłania pasujące kamienie ze swojej ręki, po czym odrzuca
kamień ze swojej ręki i zabiera skradzony kamień.
Kolejność dwóch ostatnich czynności nie jest istotna: gracz może najpierw zabrać
odrzucony kamień, a dopiero potem wykonać odrzut.

\subsubsection{Otwarta czwórka}
Kradzież ostatnio odrzuconego kamienia do czwórki rozpoczyna się od wyraźnego
powiedzenia na głos ,,kong'' lub ,,kan'' i położenia kamienia awersem do góry
razem z pasującymi trzema kamieniami z ręki gracza.
Po odsłonięciu wskaźnika \texttt{kandora}, gracz dobiera kamień zastępczy z martwego muru
i kontynuuje swoją turę tak, jak gdyby dobrał kamień z muru.

Martwy mur zawsze składa się z 14 kamieni, zatem po każdym zameldowaniu czwórki
ostatni kamień muru staje się częścią martego muru.

\subsubsection{Rozszerzenie otwartej trójki do czwórki}
Otwarta trójka może zostać rozszerzona do czwórki podczas tury gracza po tym,
jak dobrał kamień z muru lub kamień zastępczy z martwego muru (tj. nie w turze,
w której ukradł kamień do trójki lub sekwensu).
Gracz musi powiedzieć na głos ,,kong'' lub ,,kan'' i położyć czwarty kamień
równolegle do obróconego kamienia z rozszerzanej trójki.
Następnie następuje odsłonięcie wskaźnika \texttt{kandora} i dobranie kamienia
zastępczego.
Kamień użyty do rozszerzenia trójki liczy się jako odrzut i można na nim
zadeklarować wygraną.
Martwy mur zawsze składa się z 14 kamieni, zatem po każdym zameldowaniu czwórki
ostatni kamień muru staje się częścią martego muru.

\subsubsection{Zamknięta czwórka}
Zamknięta czwórka może być zadeklarowana podczas tury gracza po tym, jak dobrał
kamień z muru lub kamień zastępczy z martwego muru (tj. nie w turze, w której
ukradł kamień do trójki lub sekwensu).
Gracz musi powiedzieć na głos ,,kong'' lub ,,kan'', odsłonić kamienie składające
się na czwórkę, obrócić dwa środkowe kamienie rewersem do góry, odsłonić
wskaźnik \texttt{kandora} i dobrać kamień zastępczy.
Martwy mur zawsze składa się z 14 kamieni, zatem po każdym zameldowaniu czwórki
ostatni kamień muru staje się częścią martego muru.

Jeśli gracz nie miał otwartych grup, po zameldowaniu zamkniętej czwórki jego
ręka pozostaje zamknięta.

Na kamieniach z zamkniętej czwórki nie można zadeklarować wygranej, z wyjątkiem
wygranej na \texttt{kokushi musou}.

Cztery identyczne kamienie uznaje się za czwórkę wyłącznie wtedy, gdy nastąpiła
deklaracja czwórki.

\begin{figure}[h]
    \centering
    \begin{tikzpicture}
    \tikzset{every node/.style={font=\scriptsize}};
    \node[inner sep=0pt] at (0,0)
        {\includesvg[width=.7\linewidth]{images/dead-wall.svg}};
    \node[anchor=west,align=left] (repl) at (-5, -2.2)
        {Kolejne kamienie\\zastępcze};
    \draw[->,thick] (repl.north) -> (-1,-1);
    \node[align=center] (kandora) at (0, -2.2)
        {Kolejne wskaźniki\\\texttt{kandora}};
    \draw[->,thick] (kandora.north) -> (0.7,-0.9);
    \node[anchor=east,align=right] (wall) at (5, -2.2)
        {Kamienie z muru\\dodawane kolejno\\do martwego muru};
    \draw[->,thick] (wall.north) -> (2.75,-1);
\end{tikzpicture}

    \caption{Szczegółowy opis roli kamieni w martwym murze. Liczba na kamieniu
    odpowiada numerowi czwórki deklarowanej w trakcie rozdania.}
    \label{fig:dead-wall}
\end{figure}

\subsubsection{Pokazywanie grup}
Kamienie w zameldowanych grupach nie mogą być użyte do zbudowania innych grup
ani nie mogą być odrzucone.

Po kradzieży kamienia należy natychmiast odsłonić stosowne kamienie na ręce.
Dozwolone jest wykonanie odrzutu przed zabraniem ukradzionego kamienia.
Jeżeli ukradziony kamień nie zostanie zabrany w ciągu następnych dwóch tur
przeciwnych graczy (tj. zanim zostaną wykonane dwa odrzuty), ręka gracza staje
się martwa.

Zameldowane grupy gracz umieszcza po swojej prawej stronie, w miejscu widocznym
dla wszystkich graczy.
Ukradzione kamienie obraca się w celu zaznaczenia, który z graczy odrzucił
ukradziony kamień:
\begin{itemize}
    \item Jeśli kamień odrzucił gracz po lewej, ukradziony kamień umieszcza się
        z lewej strony grupy.
        \begin{center}
            \DrawHand{p8*79}{3em}
        \end{center}
    \item Jeśli kamień odrzucił gracz naprzeciwko, ukradziony kamień umieszcza
        się w środku grupy.
        \begin{center}
            \DrawHand{z44*4}{3em}
        \end{center}
    \item Jeśli kamień odrzucił gracz po prawej, ukradziony kamień umieszcza się
        z prawej strony grupy.
        \begin{center}
            \DrawHand{m6666*}{3em}
        \end{center}
\end{itemize}

Otwarta czwórka ma jeden obrócony kamień.
Otwarta trójka rozszerzona do czwórki ma dwa obrócone kamienie: kamień
rozszerzający umieszcza się bezpośrednio obok poprzednio obróconego.

\subsubsection{Reguła odpowiedzialności: trzecia zameldowana grupa smoków/czwarta
zameldowana grupa wiatrów}
Gracz, którego odrzucony kamień zostanie skradziony w celu sformowania:
\begin{itemize}
    \item trzeciej trójki/czwórki smoków przez gracza, który miał już dwie
        zameldowane trójki/czwórki smoków,
    \item czwartej trójki/czwórki wiatrów przez gracza, który miał już trzy
        zameldowane trójki/czwórki wiatrów,
\end{itemize}
płaci pełną wartość ręki, jeśli gracz kradnący wygra później przez dobór na
\texttt{daisangen} lub \texttt{daisuushii} (pozostali gracze nie płacą nic).
Jeśli gracz kradnący wygra przez odrzut innego gracza na \texttt{daisangen} lub
\texttt{daisuushii}, gracz, który umożliwił zameldowanie grupy i gracz, na
którego odrzucie zadeklarowano wygraną, płacą po połowie wartości ręki.
Koszt \texttt{honba} ponosi gracz, na którego odrzucie zadeklarowano wygraną.

\subsubsection{Czwarta czwórka}
Po zadeklarowaniu czwartej czwórki w trakcie rozdania rozdanie toczy się dalej,
ale nie jest dozwolone deklarowanie więcej czwórek.
Nie są przewidziane żadne wyjątki pozwalające na deklarację piątej czwórki.

\subsubsection{Wygrana na odrzucie (\texttt{ron})}
Gracz, który nie jest \texttt{furiten} i jest w stanie, używając ostatnio
odrzucony kamień, uformować poprawną mahjongową rękę z przynajmniej jednym
\texttt{yaku}, może wygrać rozdanie wyraźnie mówiąc na głos ,,ron'' lub
,,mahjong''.

\subsubsection{Wygrana na doborze (\texttt{tsumo})}
Gracz, który po doborze kamienia z muru lub kamienia zastępczego z martwego muru
jest w stanie uformować poprawną mahjongową rękę z przynajmniej jednym
\texttt{yaku}, może wygrać rozdanie wyraźnie mówiąc na głos ,,tsumo'' lub
,,mahjong''.
Gracz powinien trzymać dobrany kamień oddzielnie od reszty ręki tak, aby było
widoczne, na którym kamieniu wygrał.
Gracz, który jest \texttt{furiten}, może wygrać na doborze.

\subsubsection{\texttt{Tenpai}}
Ręka gracza jest \texttt{tenpai} (gotowa/oczekująca), jeżeli potrzebuje jednego
kamienia do skompletowania wygrywającej ręki.
Gracz jest \texttt{tenpai} również wtedy, gdy wszystkie egzemplarze kamieni, na
które czeka, są widoczne wśród odrzutów oraz zadeklarowanych grup.
Gracz nie jest \texttt{tenpai}, jeżeli czeka wyłącznie na kamień, którego
posiada wszystkie cztery egzemplarze.
Gracz z martwą ręką nie może być \texttt{tenpai}.

\subsubsection{\texttt{Riichi}}
Gracz z zamkniętą gotową ręką może zadeklarować \texttt{riichi} poprzez wyraźne
powiedzenie ,,riichi'', obrócenie na bok odrzuconego kamienia oraz zapłacenie
1,000 punktów (\emph{zakładu}) kładąc stosowny liczman przy swoich odrzutach.
Jeżeli przeciwnik zadeklaruje wygraną na obróconym kamieniu, deklaracja
\texttt{riichi} jest nieważna i zakład wraca do gracza.
Jeżeli przeciwnik ukradnie obrócony kamień do grupy, gracz powinien obrócić
następny odrzucony przez siebie kamień w swojej najbliższej turze.

Gracz nie może zadeklarować \texttt{riichi}, jeśli w murze zostały mniej niż
cztery kamienie.

Zakład wraca do gracza deklarującego \texttt{riichi}, jeżeli wygra on rozdanie.
Jeżeli inny gracz wygra rozdanie, zabiera on wniesiony zakład.
Jeżeli rozdanie wygrało kilku graczy, zakład zabiera ten gracz, który jest
najbliżej w kolejności rozgrywki (przeciwnie do ruchu wskazówek zegara) gracza,
który właśnie odrzucił kamień.
Jeżeli gra zakończy się remisem, zakład zostaje na stole aż do momentu, w którym
któryś z graczy wygra rozdanie, zabierając jednocześnie zakład.

Gracz, który zadeklarował \texttt{riichi}, nie może zmienić swojej ręki.
Może zadeklarować zamkniętą czwórkę po dobraniu kamienia pasującego do
zamkniętej trójki, ale tylko wtedy, gdy nie zmieni to kształtu jego czekania
oraz jeżeli pasujące trzy kamienie na jego ręce mogły być zinterpretowane jedynie jako
składające się na trójkę\footnote{Przykładowo, w przypadku trzech kolejnych
liczbowych trójek, gracz nie może zadeklarować żadnej zamkniętej czwórki, gdyż
ten kształt można również zinterpretować jako trzy identyczne sekwensy.}.

Gracz będący \texttt{furiten} może zadeklarować \texttt{riichi}.
Jeżeli po deklaracji \texttt{riichi} gracz nie zadeklaruje wygranej na odrzucie
kompletującym jego rękę, staje się \texttt{furiten}.
Gracz będący \texttt{furiten} może wciąż wygrać na doborze.

\texttt{Riichi} jest \texttt{yaku}.
Gracz, który wygrał podczas pierwszej sekwencji tur po złożeniu deklaracji
(wliczając w to swój dobór), otrzymuje dodatkowe \texttt{yaku} za
\texttt{ippatsu}.
Szansa na \texttt{ippatsu} przepada, jeśli sekwencja jest przerwana deklaracją
czwórki, trójki lub sekwensu, wliczając w to zamknięte czwórki.

Gracz, który wygrał po deklaracji \texttt{riichi}, odsłania kamienie z martwego
muru znajdujące się pod wskaźnikiem \texttt{dora} oraz odsłoniętymi wskaźnikami
\texttt{kandora}.
Te kamienie służą za wskaźniki \texttt{uradora} i są przyznawane wyłącznie
graczom, którzy zgłosili \texttt{riichi}.
