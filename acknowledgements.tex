\section*{Podziękowania}
\addcontentsline{toc}{section}{\protect\numberline{}Podziękowania}

Chciałabym podziękować Sjefowi Strikowi, Jenn Barr i Benjaminowi Boasowi za nieocenioną pomoc w wyjaśnianiu istoty różnic między zestawami reguł, a także za bezcenne porady.

W tej najnowszej wersji dziękuję również Sylvainowi Malbecowi za jego wkład w analizę zasad Riichi stosowanych na świecie oraz skomponowanie reguł Międzynarodowych Mistrzostw Riichi.
Za kulisami Mistrzostw miało miejsce wiele cennych dyskusji, w których brałam udział wraz z Sylvainem Malbeciem, Scottem Millerem, Jenn Barr, Gemmą Collinge oraz Garthem Nelsonem.
Jestem bardzo wdzięczna za porady oraz wgląd w świat profesjonalnych rozgrywek w Japonii uzyskany od ostatnich trzech.
Wreszcie dziękuję Komitetowi Reguł Riichi EMA (Gemma Collinge, Sven-Hendrik Gutsche, Ans Hoogland, Simon Naarman, Krzysztof Sasinowski i Alexander Wankmüller) za owocne dyskusje.
\begin{flushright}
    \emph{
        Tina Christensen\\
        Przewodnicząca Komitetu Reguł Riichi EMA\\
        Europejska Liga Mahjonga\\
        kwiecień 2016
    }
\end{flushright}