\section*{Przedmowa}
\addcontentsline{toc}{section}{\protect\numberline{}Przedmowa}

W ostatnich latach nastąpił gwałtowny i entuzjastyczny globalny rozwój madżonga.
Coraz więcej graczy z wielu krajów zaczyna doceniać walory najbardziej fascynującej gry świata.
W Europie liczba turniejów zauważalnie rośnie, tak samo zresztą jak liczba madżongowych klubów.

Dzisiejszy zglobalizowany świat z szybkim Internetem pozwala graczom z różnych krajów, a nawet kontynentów, w prosty jak nigdy sposób wspólnie zasiąść do gry przy wirtualnym stole.
Internetowe gry i dyskusje dotyczą wielu zestawów zasad, ale wariant japoński, Riichi, zawsze cieszył się szczególnym statusem oraz wyjątkowo entuzjastycznymi zwolennikami --- prawdopodobnie zainspirowanymi japońskimi profesjonalnymi graczami, jedynymi madżongowymi ,,zawodowcami'' na świecie.

Pierwsza edycja reguł Riichi według EMA została wydana podczas przygotowań do Pierwszego Europejskiego Turnieju Riichi w 2008 roku w Hanowerze (Niemcy).
Reguły opierały się na tych, które od lat były niezależnie praktykowane w Holandii i Danii.
W 2012 roku nastąpiła ich rewizja w celu przybliżenia ich do aktualnych reguł japońskiego Riichi, zachowując jednak część różnic.

Podczas przygotowań do Pierwszych Międzynarodowych Mistrzostw Riichi w Paryżu w 2014 roku powstał nowy zestaw zasad, napisany przez Sylvaina Malbeca we współpracy z wieloma osobami z zewnątrz, w tym z Profesjonalną Ligą Madżonga w Japonii.
Reguły te zostały lekko zmienione w 2015 roku i oczywistym stało się, że w świetle nowopowstałego dzieła potrzebna jest rewizja dotychczasowych zasad EMA.

Ta książeczka opisuje standardowe zasady Riichi stosowane przez Europejską Ligę Madżonga po rewizji z 2016 roku.

\begin{flushright}
    \emph{
        Tina Christensen\\
        Prezes Europejskiej Ligi Madżonga\\
        kwiecień 2016
    }
\end{flushright}