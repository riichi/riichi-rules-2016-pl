\newpage
\section{Kamienie}

Prezentujemy tu 34 podstawowe kamienie do mahjonga.
Pełen zestaw kamieni do mahjonga zawiera cztery identyczne sztuki każdego kamienia.

\subsection{Trzy rodziny}

Są trzy rodziny kamieni.
Każda z nich ma kamienie ponumerowane od jednego do dziewięciu:

\begin{center}
    \begin{tabular}{ll}
        \raisebox{1.2em}{\texttt{manzu} (liczby)}  & \DrawHand{m1~2~3~4~5~6~7~8~9}{3em} \\
        \raisebox{1.2em}{\texttt{pinzu} (monety)}  & \DrawHand{p1~2~3~4~5~6~7~8~9}{3em} \\
        \raisebox{1.2em}{\texttt{souzu} (bambusy)} & \DrawHand{s1~2~3~4~5~6~7~8~9}{3em}
    \end{tabular}
\end{center}

Jedynka bambusów jest zwykle dekorowana ptakiem, którego wzór często jest różny w różnych zestawach kamieni.
Jedynki i dziewiątki zwane są kamieniami skrajnymi (\emph{terminalami}).

\subsection{Honory}

Oprócz kamieni z poszczególnych trzech rodzin, w mahjongu występuje siedem kamieni zwanych \emph{honorami}: cztery wiatry i trzy smoki.
Wiatry są pokazane w kolejności \textbf{wschód--południe--zachód--północ}.
Smoki są pokazane w kolejności: \textbf{biały--zielony--czerwony}.
Wzór białego smoka często jest różny w różnych zestawach; zwykle jest przedstawiony jako pusty kamień lub jako niebieska prostokątna ramka.

\begin{center}
    \begin{tabular}{ll}
        \raisebox{1.2em}{\texttt{kazehai} (wiatry)}  & \DrawHand{z1~2~3~4}{3em} \\
        \raisebox{1.2em}{\texttt{sangenpai} (smoki)} & \DrawHand{z5~6~7}{3em}
    \end{tabular}
\end{center}

\subsection{Dodatkowe kamienie}

Przy czterech sztukach każdego z wyżej wymienionych kamieni, zestaw do mahjonga składa się łącznie ze 136 kamieni.
Czasami zestawy do mahjonga zawierają również inne kamienie: kwiaty, pory roku, lub jokery, które nie są jednak używane w riichi mahjongu.

Japońskie zestawy często zawierają czerwone piątki (\texttt{akapai}).
Czerwone piątki są czasem używane w zastępstwie normalnych piątek w taki sposób, że każda rodzina kamieni zawiera jedną czerwoną piątkę i trzy zwykłe piątki.
Każda czerwona piątka w ręce podnosi jej wartość o 1 \texttt{han}.
Czerwone piątki nie są już używane w zasadach riichi EMA.

\subsection{Dodatkowe wyposażenie}

Zestawy do mahjonga często posiadają znaczniki do wskazywania wiatru stołu i liczmany używane do prowadzenia punktacji.
Te ostatnie są używane również jako liczniki (\texttt{honba}) i zastawy do \texttt{riichi}.
Wartości pokazanych poniżej liczmanów to odpowiednio 100, 1,000, 5,000, i 10,000.
Dodatkowo można użyć liczmanów w innym kolorze do wyznaczania wartości 500.
Gracze rozpoczynają z 30,000 punktów.
Zestaw powinien również zawierać co najmniej dwie kości.

\makeatletter
\definecolor{myred}{HTML}{D4213D}
\newcommand\@blackbigdot[1]{\filldraw[black] #1 circle (0.16);}
\newcommand\@redbigdot[1]{\filldraw[myred] #1 circle (0.16);}
\newcommand\@blacksmalldot[1]{\filldraw[black] #1 circle (0.09);}
\newcommand\@redsmalldot[1]{\filldraw[myred] #1 circle (0.09);}
\newcommand\@drawstick[2]{
    \begin{tikzpicture}[x=#1,y=#1]
        \draw[rounded corners=5,thick] (0,0) rectangle (8,1);
        #2
    \end{tikzpicture}
}
\newcommand\StickA[1]{\@drawstick{#1}{
    \@blacksmalldot{(3.4,0.3)}
    \@blacksmalldot{(3.8,0.3)}
    \@blacksmalldot{(4.2,0.3)}
    \@blacksmalldot{(4.6,0.3)}
    \@blacksmalldot{(3.4,0.7)}
    \@blacksmalldot{(3.8,0.7)}
    \@blacksmalldot{(4.2,0.7)}
    \@blacksmalldot{(4.6,0.7)}
}}
\newcommand\StickB[1]{\@drawstick{#1}{
    \@redbigdot{(4,0.5)}
}}
\newcommand\StickC[1]{\@drawstick{#1}{
    \@redbigdot{(4,0.5)}
    \@redsmalldot{(3.6,0.25)}
    \@redsmalldot{(3.6,0.75)}
    \@redsmalldot{(4.4,0.25)}
    \@redsmalldot{(4.4,0.75)}
}}
\newcommand\StickD[1]{\@drawstick{#1}{
    \@redbigdot{(4,0.5)}
    \@redsmalldot{(3.6,0.25)}
    \@redsmalldot{(3.6,0.75)}
    \@redsmalldot{(4.4,0.25)}
    \@redsmalldot{(4.4,0.75)}
    \@redsmalldot{(3.2,0.5)}
    \@redsmalldot{(4.8,0.5)}
    \@blackbigdot{(2.5,0.5)}
    \@blackbigdot{(5.5,0.5)}
}}
\makeatother


\begin{center}
    \begin{tabular}{ll}
        \raisebox{.7em}{100} & \StickA{2em} \\
        \raisebox{.7em}{1,000} & \StickB{2em} \\
        \raisebox{.7em}{5,000} & \StickC{2em} \\
        \raisebox{.7em}{10,000} & \StickD{2em}
    \end{tabular}
\end{center}