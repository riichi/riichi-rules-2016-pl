\section{Kamienie}
\addcontentsline{toc}{section}{\protect\numberline{}Kamienie}

Prezentujemy tu 34 podstawowe kamienie do mahjonga. Pełen zestaw kamieni do mahjonga zawiera cztery identyczne sztuki każdego kamienia.

\subsection{Trzy rodziny}

Są trzy rodziny kamieni. Każda z nich ma kamienie ponumerowane od jednego do dziewięciu:

\begin{itemize}
  \item\texttt{manzu} (liczby): \DrawHand{m1~2~3~4~5~6~7~8~9}{3em}
  \item\texttt{pinzu} (monety): \DrawHand{p1~2~3~4~5~6~7~8~9}{3em}
  \item\texttt{souzu} (bambusy): \DrawHand{s1~2~3~4~5~6~7~8~9}{3em}
\end{itemize}

Jedynka bambusów jest zwykle dekorowana ptakiem, którego wzór często jest różny w różnych zestawach kamieni. Jedynki i dziewiątki zwane są kamieniami skrajnymi (terminalami).

\subsection{Honory}

Oprócz kamieni z poszczególnych trzech rodzin, w mahjongu występuje siedem kamieni zwanych honorami: cztery wiatry i trzy smoki. Wiatry są pokazane w kolejności wschód--południe--zachód--północ. Smoki są pokazane w kolejności: biały--zielony--czerwony. Wzór białego smoka często jest różny w różnych zestawach; zwykle jest przedstawiony jako pusty kamień lub jako niebieska prostokątna ramka.

\begin{itemize}
\item\texttt{kazehai} (wiatry): \DrawHand{z1~2~3~4}{3em}
\item\texttt{sangenpai} (smoki): \DrawHand{z5~6~7}{3em}
\end{itemize}

\subsection{Dodatkowe kamienie}

Przy czterech sztukach każdego z wyżej wymienionych kamieni, zestaw do mahjonga składa się z 136 kamieni w sumie. Czasami zestawy do mahjonga zawierają również inne kamienie: kwiaty, pory roku, lub jokery, które nie są używane w riichi mahjongu.

Japońskie zestawy często zawierają czerwone piątki (\texttt{akapai}). Czerwone piątki są czasem używane w zastępstwie normalnych piątek w taki sposób, że każda rodzina kamieni zawiera jedną czerwoną piątkę i trzy zwykłę piątki. Każda czerwona piątka w ręce podnosi jej wartość o 1 han. Czerwone piątki nie są już używane w zasadach EMA riichi.

\subsection{Dodatkowe wyposażenie}

Zestawy do mahjonga często posiadają znaczniki do wskazywania wiatru stołu i liczmany używane do prowadzenia punktacji. Te ostatnie są używane również jako liczniki (\texttt{honba}) i zastawy do riichi. Wartości pokazanych poniżej liczmanów to: 100, 1000, 5000, i 10000. Liczman w innym kolorze może być używany do wyznaczania wartości 500. Gracze rozpoczynają z 30000 punktów. Zestaw powinien również zawierać co najmniej dwie kości.

\begin{center}
  \includesvg{images/riichi-sticks.svg}
\end{center}

%% TODO: wstaw patyczki maciągowe.