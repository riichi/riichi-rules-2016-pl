\newpage
\section{Przygotowanie gry}

\subsection{Wiatr gracza}
\begin{figure}[h]
    \centering
    \includesvg[width=0.7\linewidth]{images/walls-winds.svg}
    \caption{Wiatry graczy przy stole od strony gracza Północy.}
\end{figure}
Mahjong jest grą czeroosobową, w której każdy z graczy ma przypisany jeden z
czterech wiatrów, nazywany \emph{wiatrem gracza}.
Grę rozpoczyna gracz Wschodu --- po jego prawej stronie siedzi gracz Południa,
naprzeciwko --- gracz Zachodu, po lewej --- gracz Północy.
W związku z tym kolejność wiatrów przy stole przeciwnie do ruchu wskazówek
zegara to \textbf{Wschód-Południe-Zachód-Północ}, zatem w szczególności nie jest
to klasyczna ,,kompasowa'' kolejność.
W kolejnych rozdaniach przypisanie wiatrów do graczy będzie się zmieniać, patrz
sekcja \ref{subsec:rotacja}.
Podczas pełnej gry każdy z graczy jest graczem Wschodu przynajmniej dwukrotnie.

\subsection{Wiatr stołu}
Kiedy gra się rozpoczyna, wiatrem stołu jest wschód.
Kiedy gracz, który rozpoczynał grę jako gracz Wschodu staje się nim ponownie po
tym, jak każdy z pozostałych graczy był graczem Wschodu przynajmniej raz,
rozpoczyna się \emph{runda południa}, tzn. wiatr stołu zmienia się na południe.
Podczas gry przy graczu, który rozpoczynał jako gracz Wschodu, powinien
znajdować się znacznik rundy pokazujący aktualny wiatr stołu.
Znacznik ten należy odwrócić wraz z rozpoczęciem rundy południa.

\subsection{Zajmowanie miejsc przy stole}
Miejsca graczy przy stole, o ile nie są ustalone z góry przez organizatorów
turnieju, są wybierane w drodze losowania.
W tym celu używa się czterech kamieni, po jednym każdego wiatru.
Jeden z graczy miesza kamienie i kładzie je rewersem do góry, po czym gracze
wybierają po jednym z wymieszanych kamieni (gracz mieszający wybiera
kamień jako ostatni).
Wybrany kamień odpowiada wiatrowi gracza na początku gry, tj. gracz, który
wybrał kamień ze wschodem, zostaje graczem Wschodu, gracz z kamieniem z
południem zostaje graczem Południa itd.

\subsection{Budowanie muru}
Kamienie należy dokładnie wymieszać.
Gracze powinni postarać się, by podczas mieszania awersy kamieni nie były
widoczne.
Następnie każdy z graczy buduje przed sobą \emph{mur} z kamieni skierowanych
awersami do dołu, długi na siedemnaście kamieni i składający się z dwóch warstw.
Mury graczy są następnie tak zsunięte do środka stołu, by tworzyły kwadrat.

\subsection{Rozbijanie muru}
\begin{figure}[h]
    \centering
    \includesvg[width=0.7\linewidth]{images/wall-break.svg}
    \caption{Rozbity mur od strony gracza Północy, gdy gracz Wschodu wyrzucił 12
    oczek.}
    \label{fig:rozbijanie}
\end{figure}
Gracz Wschodu rzuca dwiema kośćmi i odlicza graczy przeciwnie do ruchu wskazówek
zegara, zaczynając od siebie i odliczając tyle, ile wynosi suma oczek na
kościach.
Wybrany w ten sposób gracz dzieli znajdujący się przed nim mur na dwie części
--- lewą i prawą tak, by prawa część miała długość równą liczbie oczek
wyrzuconych przez gracza Wschodu (patrz Rysunek \ref{fig:rozbijanie}).

\subsection{Martwy mur}
Pierwsze siedem par kamieni w murze na prawo od miejsca rozbicia tworzy \emph{martwy
mur}.
W przypadku, gdy gracz rozbijający nie ma dość kamieni, by go uformować (tj. gdy
gracz Wschodu wyrzucił mniej niż siedem oczek), brakujący fragment dodaje się z
sąsiadującego fragmentu muru gracza po prawej stronie.
Następnie martwy mur można nieco odsunąć, aby odseparować go od pozostałych
kamieni.
Kamienie z martwego muru nie są używane w grze z wyjątkiem sytuacji, gdy któryś
z graczy zamelduje \texttt{kan}, gdy to pełnią rolę \emph{kamieni zastępczych}.

Zaleca się, aby gracz, który ma przed sobą martwy mur, położył pierwszy kamień
zastępczy bezpośrednio po lewej stronie martwego muru tak, aby martwy mur
tworzył kształt złożony z dwóch pojedynczych kamieni i sześciu par kamieni
leżących jeden na drugim.
Ma to na celu zmniejszenie ryzyka przypadkowego strącenia i odsłonięcia
pierwszego kamienia zastępczego.

\subsection{Wskaźnik \texttt{dora}}
W górnym rzędzie martwego muru należy odsłonić trzeci kamień od lewej.
Kamień ten pełni rolę \emph{wskaźnika \texttt{dora}}, tzn. określa, jaki kamień
jest \texttt{dora}:
\begin{itemize}
    \item Jeśli wskaźnik jest jednym z kamieni liczbowych, \texttt{dora} będą
        kamienie z tej samej rodziny o numerze o jeden większym (gdy wskaźnik
        jest dziewiątką, \texttt{dora} będą jedynki),
    \item Jeśli wskaźnik jest wiatrem, \texttt{dora} będą również wiatry zgodnie
        z regułą: wschód wskazuje na południe, południe --- na zachód, zachód
        --- na północ, północ --- na wschód,
    \item Jeśli wskaźnik jest smokiem, \texttt{dora} będą również smoki zgodnie
        z regułą: biały wskazuje na zielonego, zielony --- na czerwonego,
        czerwony --- na białego.
\end{itemize}

\begin{center}
    \begin{tabular}{cc}
        Wskaźnik & \texttt{dora} \\
        \DrawHand{z??m2z????}{3em} & \DrawHand{m3}{3em} \\
        \DrawHand{z??p9z????}{3em} & \DrawHand{p1}{3em} \\
        \DrawHand{z??z6z????}{3em} & \DrawHand{z7}{3em} \\
        \DrawHand{z??z4z????}{3em} & \DrawHand{z1}{3em} \\
    \end{tabular}
\end{center}

\subsection{Rozdanie}
Gracz Wschodu zabiera dla siebie pierwsze cztery kamienie (dwa górne plus dwa
dolne) znajdujące się bezpośrednio za miejscem rozbicia muru.
Następnie kamienie zabiera się czwórkami zgodnie z ruchem wskazówek zegara,
jednak gracze robią to w kolejności przeciwnej do ruchu wskazówek zegara.
To znaczy, że gracz Południa zabiera następne cztery kamienie, po nim cztery
kamienie zabiera gracz Zachodu i tak dalej, powtarzając aż do momentu, w którym
wszyscy gracze mają po dwanaście kamieni.
Następnie gracz Wschodu zabiera dwa kamienie: pierwszy i trzeci z górnego rzędu
muru (patrz Rysunek \ref{fig:chinski-szpon}), po czym gracze Południa, Zachodu i
Północy zabierają kolejno po jednym kamieniu\footnote{
    Ta sekwencja dobrań odpowiada takiej, w której wszyscy gracze dobierają
    kolejno trzynasty kamień do ręki, a następnie gracz Wschodu (jako
    rozpoczynający rozdanie) dobiera następny, czternasty.
}.
W tym momencie gracz Wschodu powinien mieć rękę składającą się z czternastu
kamieni, zaś wszyscy pozostali gracze --- z trzynastu.
\begin{figure}[h]
    \centering
    \includesvg[width=0.7\linewidth]{images/chinski-szpon.svg}
    \caption{Koniec muru zaraz po zabraniu dla siebie ostatnich kamieni przez gracza
    Wschodu.}
    \label{fig:chinski-szpon}
\end{figure}

Każdy z graczy układa kamienie w swojej ręce tak, aby były widoczne tylko dla
niego.
Gracz Wchodu zabiera kości i kładzie je po swojej prawej stronie --- stanowi to
dla wszystkich jasną informację, który z graczy jest graczem Wschodu (a zatem
również jakie są wiatry pozostałych graczy).