\newpage
\section{Przygotowanie gry}

\subsection{Wiatr gracza}
Mahjong jest grą czeroosobową, w której każdy z graczy ma przypisany jeden z
czterech wiatrów, nazywany \emph{wiatrem gracza}.
Grę rozpoczyna gracz Wschodu --- po jego prawej stronie siedzi gracz Południa,
naprzeciwko --- gracz Zachodu, po lewej --- gracz Północy.
W związku z tym kolejność wiatrów przy stole przeciwnie do ruchu wskazówek
zegara to \textbf{Wschód-Południe-Zachód-Północ}, zatem w szczególności nie jest
to klasyczna ,,kompasowa'' kolejność.
W kolejnych rozdaniach przypisanie wiatrów do graczy będzie się zmieniać, patrz
sekcja \ref{subsec:rotacja}.
Podczas pełnej gry każdy z graczy jest graczem Wschodu przynajmniej dwukrotnie.

\subsection{Wiatr stołu}
Kiedy gra się rozpoczyna, wiatrem stołu jest wschód.
Kiedy gracz, który rozpoczynał grę jako gracz Wschodu staje się nim ponownie po
tym, jak każdy z pozostałych graczy był graczem Wschodu przynajmniej raz,
rozpoczyna się \emph{runda południa}, tzn. wiatr stołu zmienia się na południe.
Podczas gry przy graczu, który rozpoczynał jako gracz Wschodu, powinien
znajdować się znacznik rundy pokazujący aktualny wiatr stołu.
Znacznik ten należy odwrócić wraz z rozpoczęciem rundy południa.

\subsection{Zajmowanie miejsc przy stole}
Miejsca graczy przy stole, o ile nie są ustalone z góry przez organizatorów
turnieju, są wybierane w drodze losowania.
W tym celu używa się czterech kamieni, po jednym każdego wiatru.
Jeden z graczy miesza kamienie i kładzie je rewersem do góry, po czym gracze
wybierają po jednym z wymieszanych kamieni (gracz mieszający wybiera
kamień jako ostatni).
Wybrany kamień odpowiada wiatrowi gracza na początku gry, tj. gracz, który
wybrał kamień ze wschodem, zostaje graczem Wschodu, gracz z kamieniem z
południem zostaje graczem Południa itd.

\subsection{Budowanie muru}
Kamienie należy dokładnie wymieszać.
Gracze powinni postarać się, by podczas mieszania awersy kamieni nie były
widoczne.
Następnie każdy z graczy buduje przed sobą \emph{mur} z kamieni skierowanych
awersami do dołu, długi na siedemnaście kamieni i składający się z dwóch warstw.
Mury graczy są następnie tak zsunięte do środka stołu, by tworzyły kwadrat.

\subsection{Rozbijanie muru}
Gracz Wschodu rzuca dwiema kośćmi i odlicza graczy przeciwnie do ruchu wskazówek
zegara, zaczynając od siebie i odliczając tyle, ile wynosi suma oczek na
kościach.
Wybrany w ten sposób gracz dzieli znajdujący się przed nim mur na dwie części
--- lewą i prawą tak, by prawa część miała długość równą liczbie oczek
wyrzuconych przez gracza Wschodu.