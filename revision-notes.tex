\section*{Wykaz zmian}
\addcontentsline{toc}{section}{\protect\numberline{}Wykaz zmian}

Reguły zostały zaktualizowane w celu usunięcia nieścisłości oraz większej zgodności z aktualnymi regułami japońskiego Riichi oraz regułami Międzynarodowych Mistrzostw Riichi.

\textbf{Główne zmiany w stosunku do wersji z 2012 roku:}
\begin{itemize}
    \item Usunięto czerwone piątki;
    \item \texttt{Tanyao} może być otwarte;
    \item \texttt{Renho} jest warte mangana, zamiast yakumana;
    \item \texttt{Daisuushii} jest warte yakumana, zamiast podwójnego yakumana;
    \item Ręka z 13+ \texttt{han} jest warta sanbaimana, zamiast yakumana;
    \item Usunięto wymóg złożenia meldunku w ciągu trzech sekund;
    \item Podmiana kamienia meldunkiem nie jest dozwolona;
    \item Tymczasowy \texttt{furiten} kończy się po doborze kamienia lub meldunku;
    \item Zniesiono wymóg posiadania dwóch \texttt{yaku} przy 5+ honbach;
    \item Usunięto reguły anulujące rozdanie w pierwszej turze (\emph{abortive draws});
    \item Usunięto \texttt{nagashi mangan};
    \item \texttt{Uma} została zmieniona na 15,000/5,000/-5,000/-15,000;
    \item \texttt{Chombo} kosztuje gracza 20,000 i jest naliczane po przyznaniu \texttt{uma};
    \item W niektórych przypadkach zostały złagodzone reguły karania graczy.
\end{itemize}

\textbf{Zmiany w stosunku do reguł Międzynarodowych Mistrzostw Riichi 2015:}
\begin{itemize}
    \item Wygrać może więcej niż jedna osoba;
    \item 4-30 nie jest zaokrąglane do mangana;
    \item Zmieniają się reguły dotyczące czasu; EMA: \texttt{pon} ma pierwszeństwo przed \texttt{chii};
    \item Zastawy pod \texttt{riichi} na końcu gry wędrują do zwycięzcy.
\end{itemize}